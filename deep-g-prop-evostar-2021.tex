\documentclass[runningheads]{llncs}
%
\usepackage{graphicx}
\usepackage{algpseudocode,algorithm}
\usepackage{hyperref}
\usepackage{listings}
\usepackage{xcolor}
\usepackage{xstring}
\usepackage[nointegrals]{wasysym}
\usepackage{MnSymbol}
\usepackage{booktabs} % For formal tables
\graphicspath{ {../images/} }

\definecolor{NavyBlue}{HTML}{000080}
\definecolor{Magenta}{HTML}{FF00FF}
\definecolor{Orange}{HTML}{FFA500}
\definecolor{CornflowerBlue}{HTML}{6495ED}
\definecolor{YellowGreen}{HTML}{9ACD32}
\definecolor{Gray}{HTML}{BEBEBE}
\definecolor{Yellow}{HTML}{FFFF00}
\definecolor{GreenYellow}{HTML}{ADFF2F}
\definecolor{ForestGreen}{HTML}{228B22}
\definecolor{Lavender}{HTML}{FFC0CB}
\definecolor{SkyBlue}{HTML}{87CEEB}
\definecolor{NavyBlue}{HTML}{000080}
\definecolor{Goldenrod}{HTML}{DDF700}
\definecolor{VioletRed}{HTML}{D02090}
\definecolor{CornflowerBlue}{HTML}{6495ED}
\definecolor{LimeGreen}{HTML}{32CD32}


% Used for displaying a sample figure. If possible, figure files should
% be included in EPS format.
%
% If you use the hyperref package, please uncomment the following line
% to display URLs in blue roman font according to Springer's eBook style:
% \renewcommand\UrlFont{\color{blue}\rmfamily}

\begin{document}
%
\title{}
%
%\titlerunning{Abbreviated paper title}
% If the paper title is too long for the running head, you can set
% an abbreviated paper title here
%
\author{First Author\inst{1}\orcidID{0000-1111-2222-3333} \and
Second Author\inst{2,3}\orcidID{1111-2222-3333-4444} \and
Third Author\inst{3}\orcidID{2222--3333-4444-5555}}
%
\authorrunning{F. Author et al.}
% First names are abbreviated in the running head.
% If there are more than two authors, 'et al.' is used.
%
\institute{Princeton University, Princeton NJ 08544, USA \and
Springer Heidelberg, Tiergartenstr. 17, 69121 Heidelberg, Germany
\email{lncs@springer.com}\\
\url{http://www.springer.com/gp/computer-science/lncs} \and
ABC Institute, Rupert-Karls-University Heidelberg, Heidelberg, Germany\\
\email{\{abc,lncs\}@uni-heidelberg.de}}
%
\maketitle              % typeset the header of the contribution
%
\begin{abstract}

\end{abstract}
%
%
%
\section{Introduction}



\section{State of the Art}
\label{soa}

% This contains a very good review. https://www.sciencedirect.com/science/article/pii/S0925231200003027?casa_token=LZtZakZguVsAAAAA:iQNvGVO5fAQO6ED-pnBhQ5Ab-ZuYP7tIdFcpBurr9hdCzNnQAKwwKUoaY6_fEbiQx5fPiktf


\section{Proposed Method}
\label{method}



\section*{Acknowledgments}

The work is
supported by the so and so.






\bibliographystyle{splncs04}
\bibliography{geneura,deep-g-prop} 

\end{document}
